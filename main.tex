\documentclass{homework}
\author{Shu Wan}
\class{CSE 691: Topics in Reinforcement Learning}
\date{\today}
\title{Homework Assignment 1}
%\address{Bayt El-Hikmah}

\graphicspath{{./media/}}

\begin{document} \maketitle

\question Write down sets in order of containment.

We pretend that equivalence classes are just numbers.
\[
	\C \supset \R \supset \Q \supset \Z \supset \N \supset
	\P \not\supset (\GF[7] = \modulo[7])  \supset \{\nil\}
\]

\question Find roots of $x^2- 8x = 9$.

We proceed by factoring,
\begin{align*}
	x^2- 8x - 9     & = 9-9         &  & \text{Subtract 9 on both sides.}         \\
	x^2- x + 9x - 9 & = 0           &  & \text{Breaking the middle term.}         \\
	(x - 1)(x + 9)  & = 0           &  & \text{Pulling out common } (x - 1).      \\
	x               & \in \{1, -9\} &  & f(x)g(x) = 0 \Ra f(x) = 0 \vee g(x) = 0. \\
\end{align*}

\fig[0.3]{cipher.png, diagram.jpg}{Cipher wheels.}{wheel}

\question Figure \ref{wheel} shows two cipher wheels. The left one is from Jeffrey Hoffstein, et al. \cite{hoffstein2008introduction} (pg. 3). Write a Python 3 program that uses it to encrypt: \texttt{FOUR SCORE AND SEVEN YEARS AGO}.

\lstinputlisting[language=Python, caption={Python 3 implementing figure \ref{wheel} left wheel.}, label=gcd]{code/prog.py}

We get: \texttt{KTZW XHTWJ FSI XJAJS DJFWX FLT}.

% citations
\bibliographystyle{plain}
\bibliography{citations}

\end{document}